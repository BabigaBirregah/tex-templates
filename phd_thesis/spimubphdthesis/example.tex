%% Use the standard UP-methodology class
%% with French language.
%%
%% You may specify the option 'twoside' or 'oneside' for
%% the document.
%%
%% See the documentation tex-upmethodology on
%% http://www.arakhne.org/tex-upmethodology/
%% for details about the macros that are provided by the class and
%% to obtain the list of the packages that are already included. 
 
\documentclass[french]{spimubphdthesis}
 
%%--------------------
%% The TeX code is entering with UTF8
%% character encoding (Linux and MacOS standards)
\usepackage[utf8]{inputenc}
 
%%-------------------
%% You want to use the NatBib extension
%\usepackage[authoryear]{natbib}
 
%%--------------------
%% Include the 'multibib' package to enable to
%% have different types of bibliographies in the
%% document (see at the end of this template for
%% an example with a personnal bibliography and
%% a general bibliography)
%%
%% Each bibliography defined with 'multibib'
%% adds a chapter with the corresponding
%% publications (in addition to the chapter for
%% the standard/general bibliography).
%% CAUTION:
%% There is no standard way to do include this type of
%% personnal bibliography.
%% We propose to use 'multibib' package to help you,
%% for example.
%\usepackage{multibib}
 
%% Define a "type" of bibliography, here the PERSONAL one,
%% that is supported by 'multibib'.
%\newcites{PERSO}{Liste de mes publications}
 
%% To cite one of your PERSONAL papers with the style
%% of the PERSONAL bibliography: \citePERSO{key}
%% To force to show one of your PERSONAL papers into
%% the PERSONAL bibliography, even if not cited in the
%% text: \nocitePERSO{key}
 
%% REMARK: When you are using 'multibib', you
%% must compile the PERSONAL bibliography by hand.
%% For example, the sequence of commands to run
%% when you had defined the bibliography PERSO is:
%%   $ pdflatex my_document.tex
%%   $ bibtex my_document.aux
%%   $ bibtex PERSO.aux
%%   $ pdflatex my_document.tex
%%   $ pdflatex my_document.tex
%%   $ pdflatex my_document.tex
 
%%--------------------
%% Add here any other packages that are needed for your document.
%\usepackage{eurosim}
%\usepackage{amsmath}
 
%%--------------------
%% Set the title, subtitle, defense date, and
%% the registration number of the PhD thesis.
%% The optional parameter is the subtitle of the PhD thesis.
%% The first mandatory parameter is the title of the PhD thesis.
%% The second mandatory parameter is the date of the PhD defense.
%% The third mandatory parameter is the reference number given by
%% the University Library after the PhD defense.
\declarethesis[Sous-titre]{Titre}{17 septembre 2012}{XXX}
 
%%--------------------
%% Set the author of the PhD thesis
\addauthor[email]{Prénom}{Nom}
 
%%--------------------
%% Add a member of the jury
%% \addjury{Firstname}{Lastname}{Role in the jury}{Position}
\addjury{Incroyable}{Hulk}{Rapporteur}{Professeur à l'Université de Gotham City \\ Commentaire secondaire}
\addjury{Super}{Man}{Examinateur}{Professeur à l'Université de Gotham City}
\addjury{Bat}{Man}{Directeur de thèse}{Professeur à l'Université de Gotham City}
 
%%--------------------
%% Change style of the table of the jury
%% \Set{jurystyle}{put macros for the style}
%\Set{jurystyle}{\small}

%%--------------------
%% Add the laboratory where the thesis was made
%\addlaboratory{Laboratoire Waynes Industry}

%%--------------------
%% Clear the list of the laboratories
%\resetlaboratories
 
%%--------------------
%% Set the English abstract
\thesisabstract[english]{This is the abstract in English}
 
%%--------------------
%% Set the English keywords. They only appear if
%% there is an English abstract
\thesiskeywords[english]{Keyword 1, Keyword 2}
 
%%--------------------
%% Set the French abstract
\thesisabstract[french]{Ceci est le résumé en français}
 
%%--------------------
%% Set the French keywords. They only appear if
%% there is an French abstract
\thesiskeywords[french]{Mot-cl\'e 1, Mot-cl\'e 2}
 
%%--------------------
%% Change the layout and the style of the text of the "primary" abstract.
%% If your document is written in French, the primary abstract is in French,
%% otherwise it is in English.
%\Set{primaryabstractstyle}{\tiny}
 
%%--------------------
%% Change the layout and the style of the text of the "secondary" abstract.
%% If your document is written in French, the secondary abstract is in English,
%% otherwise it is in French.
%\Set{secondaryabstractstyle}{\tiny}
 
%%--------------------
%% Change the layout and the style of the text of the "primary" keywords.
%% If your document is written in French, the primary keywords are in French,
%% otherwise they are in English.
%\Set{primarykeywordstyle}{\tiny}
 
%%--------------------
%% Change the layout and the style of the text of the "secondary" keywords.
%% If your document is written in French, the secondary keywords are in English,
%% otherwise they are in French.
%\Set{secondarykeywordstyle}{\tiny}
 
%%--------------------
%% Change the speciality of the PhD thesis
%\Set{speciality}{Informatique}
 
%%--------------------
%% Change the institution
%\Set{universityname}{Universit\'e de Bourgogne}
 
%%--------------------
%% Add the logos of the partners or the sponsors on the front page
%\addpartner[image options]{image name}

%%--------------------
%% Clear the list of the partner/sponsor logos
%\resetpartners

%%--------------------
%% Change the header and the foot of the pages.
%% You must include the package "fancyhdr" to
%% have access to these macros.
%% Left header
%\lhead{}
%% Center header
%\chead{}
%% Right header
%\rhead{}
%% Left footer
%\lfoot{}
%% Center footer
%\cfoot{}
%% Right footer
%\rfoot{}
 
%%--------------------
% Declare several theorems
\declareupmtheorem{mytheorem}{My Theorem}{List of my Theorems}

\begin{document}
 
%%--------------------
%% The following line does nothing until
%% the class option 'nofrontmatter' is given.
%\frontmatter

%%--------------------
%% The following line permits to add a chapter for "acknowledgements"
%% at the beginning of the document. This chapter has not a chapter
%% number (using the "star-ed" version of \chapter) to prevent it to
%% be in the table of contents
\chapter*{Remerciements}
 
%%--------------------
%% Include a general table of contents
\tableofcontents

%%--------------------
%% The content of the PhD thesis
\mainmatter
 
\part{Contexte et Problématiques}

\chapter{Introduction}
 
\section{Contexte}

Ce squelette décrit quelques éléments pouvant vous aider pour écrire votre ouvrage de thèse.
Un plan typique d'une thèse scientifique est également proposé.

\section{Objectifs de la thèse}

L'objectif principal de votre thèse peut être mis en avant à l'aide de l'environnement ci-dessous:

\begin{emphbox}
	Proposer un modèle qui fait quelque chose!
\end{emphbox}

\section{Plan de la thèse}

\chapter{\'Etat de l'art}

% The macro \chapterintro is formatting the following text according to the
% standard chapter introduction format.
%
% NOTE: if you have included the MINITOC package, the \minitoc is automatically added
\chapterintro

Pour vous aider à écrire votre ouvrage de thèse, un certain nombre d'outils sont décrits ci-dessous.
De nombreuses autres macros sont disponibles dans l'ensemble de paquets \LaTeX\ \texttt{tex-upmethodology}
sur lequel est basé le style de cette thèse. Citons pour exemples les environnements permettant de créer
automatiquement des sous-figures, les macros permettant de définir des sections non numérotées et présentes
dans le sommaire.

\section{Proposer une définition}

La définition~\ref{def:unethese} illustre la proposition d'une définition.

\begin{definition}[Une thèse] \label{def:unethese}
Ouvrage présenté devant un jury universitaire pour l'obtention d'un doctorat.
\end{definition}

\section{Inclure une figure}

L'inclusion d'une figure se réalise à l'aide des outils standards \LaTeX\ (environnement \texttt{figure}, \texttt{{\textbackslash}includegraphics}, etc.).

Nous proposons une macro permettant de réduire l'écriture de l'inclusion d'une figure.

\begin{verbatim}
\mfigure[position]{options}{filename}{titre}{labelid}
\end{verbatim}

Ceci est équivalent à (notez l'ajout de \texttt{fig:} comme préfix du label):
\begin{verbatim}
\begin{figure}[position]
	\begin{center}
		\includegraphics[options]{filename}
		\label{fig:labelid}
		\caption{titre}
	\end{center}
\end{figure}
\end{verbatim}

Le référencement de la figure peut être réalisé à l'aide des macros:
\begin{verbatim}
\figref{labelid}
\figpageref{labelid}
\end{verbatim}

\section{Inclure un tableau}

L'inclusion d'un tableau se réalise à l'aide des outils standards \LaTeX\ (environnement \texttt{table}, environment \texttt{tabularx}, etc.).

Nous proposons une macro permettant de réduire l'écriture de l'inclusion d'un tableau.

\begin{verbatim}
\begin{mtable}[options]{width}{nombrecolonnes}{columnspec}{title}{labelid}
	content
\end{mtable}
\end{verbatim}

Ceci est équivalent à (notez l'ajout de \texttt{tab:} comme préfix du label):
\begin{verbatim}
\begin{table}[options]
	\begin{center}
		\begin{tabularx}{width}{columnspec}
			content
		\end{tabularx}
		\label{tab:labelid}
		\caption{title}
	\end{center}
\end{table}
\end{verbatim}

Le référencement de la table peut être réalisé à l'aide des macros:
\begin{verbatim}
\tabref{labelid}
\tabpageref{labelid}
\end{verbatim}

\subsection{Exemple 1}

La table \tabref{exampletable1} est un exemple de table avec 4 colonnes, et dans laquelle un titre à été ajouté en sommet.
\begin{mtable}[ht]{.9\linewidth}{4}{|l|X|l|X|}{Titre de la table}{exampletable1}
	\tabulartitle{Un titre en sommet}
	\tabularheader{Col1}{Col2}{Col3}{Col4}
	a & b & c & d \\
	\hline
	e & f & g & h \\
\end{mtable}

\subsection{Exemple 2}

La table \tabref{exampletable2} est un exemple de table avec 5 colonnes, et dans laquelle le titre de la table a été également ajouté en sommet.
\begin{mtable}[ht]{.9\linewidth}{5}{|l|X|l|X|X|}{Titre de la table}{exampletable2}
	\captionastitle % Affiche le titre de la figure en sommet de table
	\tabularheader{Col1}{Col2}{Col3}{Col4}{Col5}
	a & b & c & d & x \\
	\hline
	e & f & g & h & z \\
\end{mtable}

\section{\'Enumération en ligne}

Vous pouvez énumérer des élements dans un paragraphe: \begin{inlineenumeration}
\item élement 1,
\item élement 2,
\item élement 3;
\end{inlineenumeration} et poursuivre votre texte.

\section{Description}

L'environnement \texttt{description} proposé par \LaTeX\ a été étendu:
\begin{description}
\item[\'Element 1] Texte 1
\item[\'Element 2] Texte 2
\item[\'Element 3] Texte 3
\end{description}

Omettre une entête d'item n'est pas un problème :
\begin{description}
\item[\'Element 1] Texte 1
\item Texte 2
\item[\'Element 3] Texte 3
\end{description}

\section{\'Enumération}

L'environnement \texttt{enumerate} proposé par \LaTeX\ a été étendu afin de profiter des avantages des environnements \texttt{enumerate} et \texttt{description} en un seul environnement \LaTeX:
\begin{enumerate}
\item[\'Element 1] Texte 1
\item[\'Element 2] Texte 2
\item[\'Element 3] Texte 3
\end{enumerate}

Vous pouvez spécifier le type d'énumération en passant en mode numérique arabe :
\begin{enumerate}[1]
\item[\'Element 1] Texte 1
\item[\'Element 2] Texte 2
\item[\'Element 3] Texte 3
\end{enumerate}

Ou en mode numérique romain :
\begin{enumerate}[i]
\item[\'Element 1] Texte 1
\item[\'Element 2] Texte 2
\item[\'Element 3] Texte 3
\end{enumerate}

Ou en mode numérique alphabétique :
\begin{enumerate}[a]
\item[\'Element 1] Texte 1
\item[\'Element 2] Texte 2
\item[\'Element 3] Texte 3
\end{enumerate}

Omettre une entête d'item n'est pas un problème :
\begin{enumerate}
\item[\'Element 1] Texte 1
\item Texte 2
\item[\'Element 3] Texte 3
\end{enumerate}

\section{Formatter le texte}

Vous pouvez placer un texte \textup{en exposant}. Vous pouvez placer un texte \textdown{en indice}.

Vous pouvez mettre en avant \emph{un texte}, ou le mettre \Emph{encore plus en avant}.

Vous pouvez formatter les noms de personnes de manière uniforme, comme par exemple \makename{Stéphane}{Galland} (d'autres macros sont disponibles).

\section{Symboles mathématiques}

\begin{itemize}
\item \R
\item \N
\item \Z
\item \Q
\item \C
\item $\powerset{a}$
\item $\sgn(a)$
\item $\min(a, b)$
\item $\max(a, b)$
\end{itemize}

\section{Théorèmes}

Vous pouvez définir votre propre environnement pour décrire un théorème, un lem, etc.
Ce type d'environnement doit être déclaré dans le préambule de votre document avec la
macro \texttt{{\textbackslash}declareupmtheorem} (voir l'exemple dans le préambule de
ce squelette).

\begin{mytheorem}[Theorem of Everything]
This is the theorem of Evereything.
\end{mytheorem}

\`A la fin de votre document, vous pourrez alors ajouter un chapitre listant les théorèmes présents dans votre document: \texttt{{\textbackslash}listofmytheorems}

\section{Conclusion}
 
%% Citation from the general bibliography
%\cite{key}
 
%% Citation from the PERSO bibliography
%\citePERSO{key}
 
\part{Contribution}

\chapter{Contribution}

\section{Introduction}

\section{Détails de la contribution}

\section{Conclusion}

\chapter{Réalisation}

\section{Introduction}

\section{Présentation de la réalisation}

\section{Résultats expérimentaux}

\section{Conclusion}

\part{Conclusion}

\chapter{Conclusion générale}
 
\section{Bilan}

\section{Perpectives}
 
%%--------------------
%% Start the end of the thesis
\backmatter
 
%%--------------------
%% Bibliography
 
%% PERSONAL BIBLIOGRAPHY (use 'multibib')
 
%% Change the style of the PERSONAL bibliography
%\bibliographystylePERSO{phdthesisapa}
 
%% Add the chapter with the PERSONAL bibliogaphy.
%% The name of the BibTeX file may be the same as
%% the one for the general bibliography.
%\bibliographyPERSO{biblio.bib}
 
%% Below, include a chapter for the GENERAL bibliography.
%% It is assumed that the standard BibTeX tool/approach
%% is used.
 
%% GENERAL BIBLIOGRAPHY
 
%% To cite one of your PERSONAL papers with the style
%% of the PERSONAL bibliography: \cite{key}
 
%% To force to show one of your PERSONAL papers into
%% the PERSONAL bibliography, even if not cited in the
%% text: \nocite{key}
 
%% The following line set the style of
%% the GENERAL bibliogaphy.
%% The "phdthesisapa" is a "apalike" style with the following
%% differences:
%% a) The titles are output with the color of the institution.
%% b) The name of the PhD thesis' author is underlined.
\bibliographystyle{phdthesisapa}
%% The following line may be used in place of the previous
%% line if you prefer "numeric" citations.
%\bibliographystyle{phdthesisnum}
 
%% Link the GENERAL bibliogaphy to a BibTeX file.
\bibliography{biblio.bib}
 
%%--------------------
%% List of figures and tables
 
%% Include a chapter with a list of all the figures.
%% In French typograhic standard, this list must be at
%% the end of the document.
\listoffigures
 
%% Include a chapter with a list of all the tables.
%% In French typograhic standard, this list must be at
%% the end of the document.
\listoftables
 
%%--------------------
%% Include a list of definitions
\listofdefinitions

%%--------------------
%% Appendixes
\appendix
\part{Annexes}
 
\chapter{Premier chapitre des annexes}

\chapter{Second chapitre des annexes}
 
\end{document}
